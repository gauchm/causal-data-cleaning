\documentclass[acmsmall, nonacm, screen]{acmart} %sigconf

\def\BibTeX{{\rm B\kern-.05em{\sc i\kern-.025em b}\kern-.08emT\kern-.1667em\lower.7ex\hbox{E}\kern-.125emX}}
    
\copyrightyear{2019}
\acmYear{2019}
\acmMonth{04}
%\setcopyright{acmlicensed}
%\acmConference[Woodstock '18]{Woodstock '18: ACM Symposium on Neural Gaze Detection}{June 03--05, 2018}{Woodstock, NY}
%\acmBooktitle{Woodstock '18: ACM Symposium on Neural Gaze Detection, June 03--05, 2018, Woodstock, NY}
%\acmPrice{15.00}
%\acmDOI{10.1145/1122445.1122456}
%\acmISBN{978-1-4503-9999-9/18/06}

%\citestyle{acmauthoryear}

\begin{document}

% The "title" command has an optional parameter, allowing the author to define a "short title" to be used in page headers.
\title{Towards Causal Inference in Data Error Explanation}

\author{Martin Gauch}
%\authornote{...}
\email{martin.gauch@uwaterloo.ca}
\affiliation{%
  \institution{University of Waterloo}
  \streetaddress{200 University Avenue West}
  \city{Waterloo}
  \state{Ontario}
  \postcode{N2L 3G1}
}

\begin{abstract}
...
\end{abstract}

% The code below is generated by the tool at http://dl.acm.org/ccs.cfm.
% Please copy and paste the code instead of the example below.
%\begin{CCSXML}
%<ccs2012>
% <concept>
%  <concept_id>10010520.10010553.10010562</concept_id>
%  <concept_desc>Computer systems organization~Embedded systems</concept_desc>
%  <concept_significance>500</concept_significance>
% </concept>
%</ccs2012>
%\end{CCSXML}

%\ccsdesc[500]{Computer systems organization~Embedded systems}

\keywords{data cleaning, causality, causal inference, observational study}

% A "teaser" image appears between the author and affiliation information and the body 
% of the document, and typically spans the page. 
%\begin{teaserfigure}
%  \includegraphics[width=\textwidth]{sampleteaser}
%  \caption{Seattle Mariners at Spring Training, 2010.}
%  \Description{Enjoying the baseball game from the third-base seats. Ichiro Suzuki preparing to bat.}
%  \label{fig:teaser}
%\end{teaserfigure}

\maketitle

\section{Introduction}

\subsection{Problem Statement}

\subsection{Challenges}

\subsection{Contributions}


\section{Background}

\subsection{Observational Studies}

\subsection{Causal Inference}

\subsection{Error Explanation}

\section{Related Work}
In the context of databases, Wang et al. use Bayesian analysis to help users finding reasons for errors in a dataset \cite{Wang15}. Their framework uses a heuristic cost function that optimizes for conciseness, specificity and consistency. Further research focuses on explaining errors in database query
output and either tries to fix the underlying data \cite{Wu13} or the query \cite{Tran10}.
Causal inference is much more exhaustively researched in social and medical sciences. In these
areas, scientists often need to test cause-effect hypotheses through observational studies, as randomized trials are not always ethically or physically feasible \cite{Rosenbaum02}. Propensity score stratification
\cite{Austin11} is one of several common techniques used in observational studies that tries to reduce any bias introduced by confounding factors when available data is limited.
With Big Data applications becoming increasingly ubiquitous, Hernán et al. study to what extent we can approximate randomized experiments through using large sets of observational data \cite{Robins16}.
While clearly not every trial can be emulated through data analysis, they find that under reasonable assumptions it is possible to approximate randomization.

\section{Causal Inference for Error Explanation}

\section{Experiments}

\section{Conclusion and Future Work}

%\begin{acks}
%...
%\end{acks}

\bibliographystyle{ACM-Reference-Format}
\bibliography{report}

\appendix

\section{Research Methods}

\subsection{Part One}
...

\end{document}
